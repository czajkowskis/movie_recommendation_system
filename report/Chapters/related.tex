In this section, we explore the previous research on different movie recommendation systems. We focus on currently researched approaches as well asulczyńska established standards.

\subsection{Collaborative Filtering}

Collaborative Filtering systems provide recommendations based on the preferences of similar users (Subramaniyaswamy et al. 2017) \cite{doi:10.1504/IJHPCN.2017.083199}. Such systems use a variety of algorithms (Ekstrand et al. 2011) \cite{HCI-009} and are often combined with Content-based approaches to enhance the predictive performance of the model (Geetha et al. 2018) \cite{Geetha_2018}.

\subsection{Content-based Filtering}
Content-based Filtering systems use specific movie features such as genre, cast or director to make predictions about user preferences. Currently reaserached approaches include using Genre Correlation (Reddy et al. 2018 \cite{10.1007/978-981-13-1927-3_42} multiattribute networks (Son \& Kim, 2017) \cite{SON2017404} and temporal user preferences (Cami et al. 2017) \cite{8311601}.

\subsection{Deep learning Approaches}
Several studies explore the usage of Deep Learning approaches in building movie recomendation systems. Proposed systems use Autoencoders (Lund \& Ng, 2018) \cite{8424686} (Hossein Tahmasebi et al. 2020) \cite{Tahmasebi2021}, Restricted Boltzmann Machines (Abdollahi \& Nasraoui, 2016) \cite{abdollahi2016explainable} and Multlayer Perceptron Networks (Jena et al. 2022) \cite{Jena2022}. Such models are used to build both systems based on Collaborative Filtering and Content-based Filtering as well as hybrid ones.